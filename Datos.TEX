%%%%%%%%%%%%%%%%%%%%%%%%%%%%% CARATULA%%%%%%%%%%%%%%%%%%%%%%%%
\textheight 19cm
\pagestyle{empty}
\begin{center}
 {\bf {\fontsize{14}{16.8}\selectfont UNIVERSIDAD NACIONAL DE TRUJILLO}}     
 
    {\bf{\fontsize{14}{16.8}\selectfont Facultad de Ciencias Físicas y Matemáticas}} 

  {\bf{\fontsize{14}{16.8}\selectfont Escuela Profesional de Informática}}
\end{center}  

\begin{figure}[ht]
\begin{center}
\includegraphics[width=.4\textwidth]{unt}
\end{center}
\end{figure}

\vskip 2cm
\begin{center}
  { \bf {\fontsize{17}{20.4}\selectfont{DISEÑO DE UN AGENTE INTELIGENTE PARA EL RECONOCIMIENTO DE EXPRESIONES FACIALES EN SECUENCIAS DINÁMICAS DE IMÁGENES}}     
  \vskip 3cm
	{\bf \fontsize{14}{16.8}\selectfont {\hspace{-2.9cm}AUTOR(ES): Carlos E. Plasencia Prado}}} \\
    {\bf \fontsize{14}{16.8}\selectfont {\hspace{-0.4cm}Piere A. Ruiz Alba}}\\
    \vskip 0.2cm
    {\bf \fontsize{14}{16.8}\selectfont {\hspace{-1.7cm} ASESOR: Juan O. Salazar Campos}}	  

\end{center}   
\vskip 1.1cm
\begin{center}    
{\bf {\fontsize{14}{16.8}\selectfont Trujillo - La Libertad
\vskip 0.0cm
\hspace*{-0.2cm} 
2019 }}
\end{center} 
\newpage
%%%%%%%%%%%%%%%%%%%%%%%%%%%%%%%%%%%%%%%%%%%%%%%%%%%%%%%%%%%%%%%%%%%%%%%%%%%


%%%%%%%%%%%%%%%%%%%%%%%%%%%%CONTRA CARATULA 1 %%%%%%%%%%%%%%%%%%%%%%%%%%%%%
\newpage
\pagestyle{plain}
\pagenumbering{roman}

\hspace*{6cm}
\vskip 9cm
\begin{center}
   {\bf \doublespacing {\fontsize{17}{20.4}\selectfont{DISEÑO DE UN AGENTE INTELIGENTE PARA EL RECONOCIMIENTO DE EXPRESIONES FACIALES EN SECUENCIAS DINÁMICAS DE IMÁGENES}}}     
\end{center} 
\newpage
%%%%%%%%%%%%%%%%%%%%%%%%%%%%%%%%%%%%%%%%%%%%%%%%%%%%%%%%%%%%%%%%%%%%%%%%%%%


%%%%%%%%%%%%%%%%%%%%%%%%%%%%% CONTRA CARATULA 2 %%%%%%%%%%%%%%%%%%%%%%%
\begin{center}
   {\bf {\fontsize{14}{16.8}\selectfont{CARLOS PLASENCIA PRADO}}}\\    
      {\bf {\fontsize{14}{16.8}\selectfont{PIERE RUIZ ALBA}}}       
   \end{center}   

\vskip 3.2cm
\begin{center}
   {\bf \doublespacing {\fontsize{17}{20.4}\selectfont{DISEÑO DE UN AGENTE INTELIGENTE PARA EL RECONOCIMIENTO DE EXPRESIONES FACIALES EN SECUENCIAS DINÁMICAS DE IMÁGENES}}}     
\end{center}   
  \vskip 2cm
\begin{verse}
 \fontsize{12}{14.4}\selectfont{\hspace*{0.6cm}Tesis presentada a la Escuela Profesional de Informática en la Facultad de Ciencias Físicas y Matemáticas de la Universidad Nacional de Trujillo, como requisito parcial para la obtención del Título profesional de Ing. Informático.}
\end{verse}

\vskip 1.5cm 
{\fontsize{14}{16.8}\selectfont ASESOR: JUAN O. SALAZAR CAMPOS} 
 \vskip 1cm 
 \begin{center}    
 \vskip 1.5cm
{\fontsize{14}{16.8}\selectfont Trujillo - La Libertad
\vskip 0.1cm
\hspace*{-0.2cm} 
2019}
\end{center} 
\newpage
%%%%%%%%%%%%%%%%%%%%%%%%%%%%%%%%%%%%%%%%%%%%%%%%%%%%%%%%%%%%%%%%%%%%%%%%%%%%%


%%%%%%%%%%%%%%%%%%%%%%%%%%%%HOJA DE APROBACION %%%%%%%%%%%%%%%%%%%%%%%%%%%%%
\begin{center}
 {\bf {\Large HOJA DE APROBACIÓN }     
 \vskip 1.5cm
  {\Large Diseño de un agente inteligente para el reconocimiento de expresiones faciales en secuencias dinámicas de imágenes}}
 \vskip 1cm 
  {\large{Carlos Eduardo Plasencia Prado}}\\
    {\large{Piere Andre Ruiz Alba}}

 \vskip 1cm
\end{center} 
Tesis defendida y aprobada por el jurado examinador:
\vskip 1.2 cm
\begin{flushleft} 
$\overline{\mbox{Prof. Ing. Juan O. Salazar Campos - Asesor}}$\\
\vskip -0.5cm
Departamento de Informática - UNT
\end{flushleft} 
\vskip 0.8cm
\begin{flushleft} 
$\overline{\mbox{Prof. Ing. Edwin R. Mendoza Torres - Presidente}}$\\
\vskip -0.5cm
Departamento de Informática - UNT
\end{flushleft} 
\vskip 0.8cm
\begin{flushleft} 
$\overline{\mbox{Prof. Ing. Iris Á. Cruz Florián - Secretaria}}$\\
\vskip -0.5cm
Departamento de Informática - UNT
\end{flushleft}
\vskip 0.5cm 
\begin{center}    
Trujillo, Junio del 2019
\end{center} 
\newpage
%%%%%%%%%%%%%%%%%%%%%%%%%%%%%%%%%%%%%%%%%%%%%%%%%%%%%%%%%%%%%%%%%%%%%%%%%%%%


%%%%%%%%%%%%%%%%%%%%%%%%%%%% DEDICATORIA %%%%%%%%%%%%%%%%%%%%%%
 
 \addcontentsline{toc}{chapter}{Dedicatoria}
 {\bf\Large {Dedicamos esta tesis a :}}
 \vskip 1cm
\begin{quotation}
{\it A Nuestros Padres por todo el amor que nos han brindado,y por todo el apoyo que nos dieron para poder llegar hasta este punto.
\vskip 1cm
A nuestros seres mas queridos por siempre estar apoyandonos y aconsejandonos.
}
\end{quotation}
%%%%%%%%%%%%%%%%%%%%%%%%%%%%%%%%%%%%%%%%%%%%%%%%%%%%%%%%%%%%%%%%%%%%%%%%%%%


%%%%%%%%%%%%%%%%%%%%%%%%%%%% AGRADECIMENTOS %%%%%%%%%%%%%%%%%%%%%%
\newpage

 \addcontentsline{toc}{chapter}{Agradecimientos}
 {\bf\Large {\flushleft{Agradecimientos}}}
 \vskip 1.5cm
\begin{quotation}
Agradecemos a Dios por habernos bendecido en toda nuestra vida.
{\vskip 1cm}
A nuestro asesor el Prof. Juan Orlando Salazar Campos que siempre se mostro disponible e interesado en ayudarnos.
\vskip 1cm
A nuestros profesores del Departamento de Informática, de los cuales recibimos una gran cantidad de conocimientos.
\vskip 1cm
 \end{quotation}
%%%%%%%%%%%%%%%%%%%%%%%%%%%%%%%%%%%%%%%%%%%%%%%%%%%%%%%%%%%%%%%%%%%%%%%%%%%


%%%%%%%%%%%%%%%%%%%%%%%%%%%% RESUMEN%%%%%%%%%%%%%%%%%%%%%%
\newpage
\begin{center}
 \addcontentsline{toc}{chapter}{Resumen}
 {\bf\LARGE Resumen}
\end{center} 
\vskip 0.5cm
\begin{quotation}

La presente tesis trata acerca de cómo reconocer expresiones faciales en secuencias dinámicas de imágenes, aunque parece simple, pero es muy difícil debido a la alta variabilidad en la secuencia de imágenes. La solución propuesta para este problema consiste en el diseño de un agente inteligente que permite el reconocimiento de expresiones faciales en secuencias dinámicas de imágenes, el agente inteligente capta la secuencia mediante una cámara, luego el agente la descompone y analiza las imágenes obtenidas, luego el preprocesamiento de imágenes permite al algoritmo “histograma de gradiente orientado” (HOG) obtener la ubicación del rostro y segmentar el rostro de la imagen para posteriormente extraer los puntos característicos o rasgos faciales con el algoritmo “patrones binarios locales” (LBP) y mediante el clasificador “máquina de vector soporte” (SVM) poder identificar las expresiones faciales: asombro, asco, enojo, felicidad, miedo, tristeza y normal, para que luego el agente muestre el nombre de la expresión reconocida en el monitor del computador. Se utilizó una población indefinida que nos dio una muestra de 349 secuencias dinámicas de imágenes de las cuales 224 se utilizaron para el entrenamiento, 55 para pruebas internas y finalmente 70 para pruebas finales, obteniendo un resultado del 93.68\% de precisión.

\vskip 0.3cm
\hspace*{-0.6cm}{\bf Palabras claves:} reconocimiento, expresiones faciales, agente inteligente.
\end{quotation}
%%%%%%%%%%%%%%%%%%%%%%%%%%%%%%%%%%%%%%%%%%%%%%%%%%%%%%%%%%%%%%%%%%%%%%%%%%%%%%%%%%%%


%%%%%%%%%%%%%%%%%%%%%%%%%%%%ABSTRACT%%%%%%%%%%%%%%%%%%%%%%
\newpage
\begin{center}
 \addcontentsline{toc}{chapter}{Abstract}
 {\bf\LARGE Abstract}\vskip 1.5cm
\end{center} 
\begin{quotation}

This thesis deals with how to recognize facial expressions in dynamic sequences of images, although it seems easy, but it is very difficult to recognize expressions due to the high variability in the sequence of images. The proposed solution for this problem consists in the design of an intelligent agent that allows the recognition of facial expressions in dynamic sequences of images, the intelligent agent captures the sequence through a camera, then decomposes the sequence into images and analyzes the images obtained, then the image preprocessing allows the algorithm "histogram oriented gradient" (HOG) to obtain the location of the face and segment the face of the image to then extract the minutiae or facial features with the algorithm "local binary patterns" (LBP) and using the classifier "support vector machine" (SVM) to identify facial expressions: amazement, disgust, anger, happiness, fear, sadness and normal, so that later the agent shows the name of the expression recognized in the monitor of the computer. We used an indefinite population that gave us a sample of 349 dynamic sequences of images of which 224 were used for training, 55 for internal tests and finally 70 for final tests, obtaining a result of 93.68\% accuracy.


\vskip 0.3cm
\hspace*{-0.6cm}{\bf Keywords:} recognition, facial expressions, intelligent agent.
\end{quotation}
%%%%%%%%%%%%%%%%%%%%%%%%%%%%%%%%%%%%%%%%%%%%%%%%%%%%%%%%%%%%%%%%%%%%%%%%%%%%%%

