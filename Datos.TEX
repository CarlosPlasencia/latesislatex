%%%%%%%%%%%%%%%%%%%%%%%%%%%%% CARATULA%%%%%%%%%%%%%%%%%%%%%%%%
\textheight 19cm
\pagestyle{empty}
\begin{center}
 {\bf {\fontsize{14}{16.8}\selectfont UNIVERSIDAD NACIONAL DE TRUJILLO}}     
 
    {\bf{\fontsize{14}{16.8}\selectfont Facultad de Ciencias Físicas y Matemáticas}} 

  {\bf{\fontsize{14}{16.8}\selectfont Escuela Profesional de Informática}}
\end{center}  

\begin{figure}[ht]
\begin{center}
\includegraphics[width=.4\textwidth]{image/unt}
\end{center}
\end{figure}

\vskip 2cm
\begin{center}
  { \bf {\fontsize{17}{20.4}\selectfont{Diseño de un agente inteligente para el reconocimiento de expresiones faciales en secuencias dinámicas de imágenes.}}     
  \vskip 3cm
  {\bf \fontsize{14}{16.8}\selectfont {Bch. Carlos Eduardo Plasencia Prado}}} \\
    {\bf \fontsize{14}{16.8}\selectfont {Bch. Piere Andre Ruiz Alba}} 

\end{center}   
\vskip 1.3cm
\begin{center}    
{\bf {\fontsize{14}{16.8}\selectfont Trujillo - La Libertad
\vskip 0.0cm
\hspace*{-0.2cm} 
2018 }}
\end{center} 
\newpage
%%%%%%%%%%%%%%%%%%%%%%%%%%%%%%%%%%%%%%%%%%%%%%%%%%%%%%%%%%%%%%%%%%%%%%%%%%%


%%%%%%%%%%%%%%%%%%%%%%%%%%%%CONTRA CARATULA 1 %%%%%%%%%%%%%%%%%%%%%%%%%%%%%
\newpage
\pagestyle{plain}
\pagenumbering{roman}

\hspace*{6cm}
\vskip 9cm
\begin{center}
   {\bf \doublespacing {\fontsize{17}{20.4}\selectfont{DISEÑO DE UN AGENTE INTELIGENTE PARA EL RECONOCIMIENTO DE EXPRESIONES FACIALES EN SECUENCIAS DINÁMICAS DE IMÁGENES }}}     
\end{center} 
\newpage
%%%%%%%%%%%%%%%%%%%%%%%%%%%%%%%%%%%%%%%%%%%%%%%%%%%%%%%%%%%%%%%%%%%%%%%%%%%


%%%%%%%%%%%%%%%%%%%%%%%%%%%%% CONTRA CARATULA 2 %%%%%%%%%%%%%%%%%%%%%%%
\begin{center}
   {\bf {\fontsize{14}{16.8}\selectfont{CARLOS EDUARDO PLASENCIA PRADO}}}\\    
      {\bf {\fontsize{14}{16.8}\selectfont{PIERE ANDRE RUIZ ALBA}}}       
   \end{center}   

\vskip 2.5cm
\begin{center}
   {\bf \doublespacing {\fontsize{17}{20.4}\selectfont{DISEÑO DE UN AGENTE INTELIGENTE PARA EL RECONOCIMIENTO DE EXPRESIONES FACIALES EN SECUENCIAS DINÁMICAS DE IMÁGENES }}}     
\end{center}   
  \vskip 2cm
\begin{verse}
 \fontsize{12}{14.4}\selectfont{\hspace*{0.6cm}Tesis presentada a la Escuela Profesional de Informática en la Facultad de Ciencias Físicas y Matemáticas de la Universidad Nacional de Trujillo, como requisito parcial para la obtención del Título profesional de Ing. Informático}
\end{verse}

\vskip 1.5cm 
{\fontsize{14}{16.8}\selectfont ASESOR: JUAN ORLANDO SALAZAR CAMPOS} 
 \vskip 1cm 
 \begin{center}    
 \vskip 2cm
{\fontsize{14}{16.8}\selectfont Trujillo - La Libertad
\vskip 0.2cm
\hspace*{-0.2cm} 
2018}
\end{center} 
\newpage
%%%%%%%%%%%%%%%%%%%%%%%%%%%%%%%%%%%%%%%%%%%%%%%%%%%%%%%%%%%%%%%%%%%%%%%%%%%%%


%%%%%%%%%%%%%%%%%%%%%%%%%%%%HOJA DE APROBACION %%%%%%%%%%%%%%%%%%%%%%%%%%%%%
\begin{center}
 {\bf {\Large HOJA DE APROBACIÓN }     
 \vskip 1cm
  {\Large Diseño de un agente inteligente para el reconocimiento de expresiones faciales en secuencias dinámicas de imágenes. }}
 \vskip 1cm 
  {\large{Carlos Eduardo Plasencia Prado}}\\
    {\large{Piere Andre Ruiz Alba}}

 \vskip 1cm
\end{center} 
Tesis defendida y aprobada por el jurado examinador:
\vskip 1.5 cm
\begin{flushleft} 
$\overline{\mbox{Prof. Juan O. Salazar Campos - Asesor}}$\\
\vskip -0.5cm
Departamento de Informática - UNT
\end{flushleft} 
\vskip 1cm
\begin{flushleft} 
$\overline{\mbox{Prof. Edwin R. Mendoza Torres.}}$\\
\vskip -0.5cm
Departamento de Informática - UNT
\end{flushleft} 
\vskip 1cm
\begin{flushleft} 
$\overline{\mbox{Prof. Iris A. Cruz Florián}}$\\
\vskip -0.5cm
Departamento de Informática - UNT
\end{flushleft}
\vskip 0.8cm 
\begin{center}    
Trujillo, 23 de diciembre del 2018
\end{center} 
\newpage
%%%%%%%%%%%%%%%%%%%%%%%%%%%%%%%%%%%%%%%%%%%%%%%%%%%%%%%%%%%%%%%%%%%%%%%%%%%%


%%%%%%%%%%%%%%%%%%%%%%%%%%%% DEDICATORIA %%%%%%%%%%%%%%%%%%%%%%
 
 \addcontentsline{toc}{chapter}{Dedicatoria}
 {\bf\Large {Dedicamos esta tesis a :}}
 \vskip 1cm
\begin{quotation}
{\it A Nuestros Padres por todo el amor que nos han brindado,y por todo el apoyar que nos dieron para poder llegar hasta este punto.
\vskip 1cm
A nuestros seres mas queridos por siempre estar apoyandonos y aconsejandonos.
}
\end{quotation}
%%%%%%%%%%%%%%%%%%%%%%%%%%%%%%%%%%%%%%%%%%%%%%%%%%%%%%%%%%%%%%%%%%%%%%%%%%%


%%%%%%%%%%%%%%%%%%%%%%%%%%%% AGRADECIMENTOS %%%%%%%%%%%%%%%%%%%%%%
\newpage

 \addcontentsline{toc}{chapter}{Agradecimientos}
 {\bf\Large {\flushleft{Agradecimientos}}}
 \vskip 1.5cm
\begin{quotation}
Agradecemos a Dios por habernos bendecido en toda nuestra vida...
{\vskip 1cm}
A nuestros profesores del Departamento de Informática, de los cuales recibimos una gran cantidad de conocimientos  . . .
\vskip 1cm
A nuestro asesor el Prof. Juan Orlando Salazar Campos que siempre se mostro disponible e interesado en ayudarnos.
\vskip 1cm
 \end{quotation}
%%%%%%%%%%%%%%%%%%%%%%%%%%%%%%%%%%%%%%%%%%%%%%%%%%%%%%%%%%%%%%%%%%%%%%%%%%%


%%%%%%%%%%%%%%%%%%%%%%%%%%%% RESUMEN%%%%%%%%%%%%%%%%%%%%%%
\newpage
\begin{center}
 \addcontentsline{toc}{chapter}{Resumen}
 {\bf\LARGE Resumen}
\end{center} 
\vskip 0.5cm
\begin{quotation}

El presente trabajo tiene como objetivo el diseño de un agente inteligente que permita el reconocimiento de expresiones faciales en secuencias dinámicas de imágenes, el agente inteligente capta las secuencias dinámicas de imágenes mediante una cámara, luego descompone la secuencia en imágenes y analiza las imágenes obtenidas para que mediante el uso de algoritmos como el realce por histograma y filtro de ruido gausiano permitir al algoritmo histograma de gradiente orientado (HOG) obtener la ubicación del rostro, después de haber obtenido las coordenadas referentes al rostro se procede a segmentar el rostro de la imagen para posteriormente extraer los puntos característicos o rasgos faciales con el algoritmo de patrones binarios locales (LBP) y mediante el clasificador máquina de vector soporte(SVM) que fue entrenado con secuencias dinámicas de imágenes de expresiones faciales, poder identificar las expresiones faciales o gestos como pueden ser : asombro, asco, enojo, felicidad, miedo, tristeza, y un estado normal, finalmente el agente muestra la expresión reconocida en el monitor del computador.
    

\vskip 0.3cm
\hspace*{-0.6cm}{\bf Palabras claves:} reconocimiento, expresiones faciales, agente inteligente.
\end{quotation}
%%%%%%%%%%%%%%%%%%%%%%%%%%%%%%%%%%%%%%%%%%%%%%%%%%%%%%%%%%%%%%%%%%%%%%%%%%%%%%%%%%%%


%%%%%%%%%%%%%%%%%%%%%%%%%%%%ABSTRACT%%%%%%%%%%%%%%%%%%%%%%
\newpage
\begin{center}
 \addcontentsline{toc}{chapter}{Abstract}
 {\bf\LARGE Abstract}\vskip 1.5cm
\end{center} 
\begin{quotation}

The present work aims at the design of an intelligent agent that allows the recognition of facial expressions in dynamic sequences of images, the intelligent agent captures the dynamic sequences of images by a camera, then decomposes the sequence into images and analyzes the images to that the use of algorithms such as histogram enhancement and the Gaussian noise filter allow the algorithm oriented gradient histogram (HOG) to obtain the location of the face, after having done it, the coordinates refer to the face, we proceed to a segment of the face of the image to later extract the characteristic points or facial features with the algorithm of local binary patterns (LBP) and by means of the support vector machine classifier (SVM) that was trained with sequences of images of facial expressions, to be able to identify the expressions Facials or gestures such as: amazement, disgust, anger, happiness, fear, tris teza, and normal state, finally the agent shows the recognized expression on the computer monitor.


\vskip 0.3cm
\hspace*{-0.6cm}{\bf Keywords:} recognition, facial expressions, intelligent agent.
\end{quotation}
%%%%%%%%%%%%%%%%%%%%%%%%%%%%%%%%%%%%%%%%%%%%%%%%%%%%%%%%%%%%%%%%%%%%%%%%%%%%%%


%%%%%%%%%%%%%%%%%%%%%%%%%%% LISTA DE SIMBOLOS %%%%%%%%%%%%%%%%%%%%%%
\newpage
\addcontentsline{toc}{chapter}{Lista de símbolos}
 {\bf\LARGE Lista de símbolos}
 \vskip 1.5cm
Constantes: 
\begin{enumerate}
\item[(1)]$r,\overline{r} $ \hspace*{0.8cm} Indice que denota regiones.
\item[(2)] $n $ \hspace*{1.1cm} Indice de bienes finales deseados por los consumidores.
\item[(3)] ...
\vskip 3cm
\end{enumerate} 
\vskip 0.3cm
Variables:
\begin{enumerate}
\item[(5)] $ x^{r} $ \hspace*{1cm} Vector columna que denota la actividad de producción.
\item[(6)] $ u^{r} $ \hspace*{1.2cm} . . .
\end{enumerate}
