%%%%%%%%%%%%%%%%%%%%%%%%%%%%% CARATULA%%%%%%%%%%%%%%%%%%%%%%%%
\textheight 19cm
\pagestyle{empty}
\begin{center}
 {\bf {\fontsize{14}{16.8}\selectfont UNIVERSIDAD NACIONAL DE TRUJILLO}}     
 
    {\bf{\fontsize{14}{16.8}\selectfont Facultad de Ciencias Físicas y Matemáticas}} 

  {\bf{\fontsize{14}{16.8}\selectfont Escuela Profesional de Informática}}
\end{center}  

\begin{figure}[ht]
\begin{center}
\includegraphics[width=.4\textwidth]{unt}
\end{center}
\end{figure}

\vskip 2cm
\begin{center}
  { \bf {\fontsize{17}{20.4}\selectfont{Diseño de un agente inteligente para el reconocimiento de expresiones faciales en secuencias dinámicas de imágenes.}}     
  \vskip 3cm
  {\bf \fontsize{14}{16.8}\selectfont {Bch. Carlos Eduardo Plasencia Prado}}} \\
    {\bf \fontsize{14}{16.8}\selectfont {Bch. Piere Andre Ruiz Alba}} 

\end{center}   
\vskip 1.3cm
\begin{center}    
{\bf {\fontsize{14}{16.8}\selectfont Trujillo - La Libertad
\vskip 0.0cm
\hspace*{-0.2cm} 
2018 }}
\end{center} 
\newpage
%%%%%%%%%%%%%%%%%%%%%%%%%%%%%%%%%%%%%%%%%%%%%%%%%%%%%%%%%%%%%%%%%%%%%%%%%%%


%%%%%%%%%%%%%%%%%%%%%%%%%%%%CONTRA CARATULA 1 %%%%%%%%%%%%%%%%%%%%%%%%%%%%%
\newpage
\pagestyle{plain}
\pagenumbering{roman}

\hspace*{6cm}
\vskip 9cm
\begin{center}
   {\bf \doublespacing {\fontsize{17}{20.4}\selectfont{DISEÑO DE UN AGENTE INTELIGENTE PARA EL RECONOCIMIENTO DE EXPRESIONES FACIALES EN SECUENCIAS DINÁMICAS DE IMÁGENES }}}     
\end{center} 
\newpage
%%%%%%%%%%%%%%%%%%%%%%%%%%%%%%%%%%%%%%%%%%%%%%%%%%%%%%%%%%%%%%%%%%%%%%%%%%%


%%%%%%%%%%%%%%%%%%%%%%%%%%%%% CONTRA CARATULA 2 %%%%%%%%%%%%%%%%%%%%%%%
\begin{center}
   {\bf {\fontsize{14}{16.8}\selectfont{CARLOS EDUARDO PLASENCIA PRADO}}}\\    
      {\bf {\fontsize{14}{16.8}\selectfont{PIERE ANDRE RUIZ ALBA}}}       
   \end{center}   

\vskip 2.5cm
\begin{center}
   {\bf \doublespacing {\fontsize{17}{20.4}\selectfont{DISEÑO DE UN AGENTE INTELIGENTE PARA EL RECONOCIMIENTO DE EXPRESIONES FACIALES EN SECUENCIAS DINÁMICAS DE IMÁGENES }}}     
\end{center}   
  \vskip 2cm
\begin{verse}
 \fontsize{12}{14.4}\selectfont{\hspace*{0.6cm}Tesis presentada a la Escuela Profesional de Informática en la Facultad de Ciencias Físicas y Matemáticas de la Universidad Nacional de Trujillo, como requisito parcial para la obtención del Título profesional de Ing. Informático}
\end{verse}

\vskip 1.5cm 
{\fontsize{14}{16.8}\selectfont ASESOR: JUAN ORLANDO SALAZAR CAMPOS} 
 \vskip 1cm 
 \begin{center}    
 \vskip 2cm
{\fontsize{14}{16.8}\selectfont Trujillo - La Libertad
\vskip 0.2cm
\hspace*{-0.2cm} 
2018}
\end{center} 
\newpage
%%%%%%%%%%%%%%%%%%%%%%%%%%%%%%%%%%%%%%%%%%%%%%%%%%%%%%%%%%%%%%%%%%%%%%%%%%%%%


%%%%%%%%%%%%%%%%%%%%%%%%%%%%HOJA DE APROBACION %%%%%%%%%%%%%%%%%%%%%%%%%%%%%
\begin{center}
 {\bf {\Large HOJA DE APROBACIÓN }     
 \vskip 1cm
  {\Large Diseño de un agente inteligente para el reconocimiento de expresiones faciales en secuencias dinámicas de imágenes. }}
 \vskip 1cm 
  {\large{Carlos Eduardo Plasencia Prado}}\\
    {\large{Piere Andre Ruiz Alba}}

 \vskip 1cm
\end{center} 
Tesis defendida y aprobada por el jurado examinador:
\vskip 1.5 cm
\begin{flushleft} 
$\overline{\mbox{Prof. Juan O. Salazar Campos - Asesor}}$\\
\vskip -0.5cm
Departamento de Informática - UNT
\end{flushleft} 
\vskip 1cm
\begin{flushleft} 
$\overline{\mbox{Prof. Edwin R. Mendoza Torres.}}$\\
\vskip -0.5cm
Departamento de Informática - UNT
\end{flushleft} 
\vskip 1cm
\begin{flushleft} 
$\overline{\mbox{Prof. Iris A. Cruz Florián}}$\\
\vskip -0.5cm
Departamento de Informática - UNT
\end{flushleft}
\vskip 0.8cm 
\begin{center}    
Trujillo, 23 de diciembre del 2018
\end{center} 
\newpage
%%%%%%%%%%%%%%%%%%%%%%%%%%%%%%%%%%%%%%%%%%%%%%%%%%%%%%%%%%%%%%%%%%%%%%%%%%%%


%%%%%%%%%%%%%%%%%%%%%%%%%%%% DEDICATORIA %%%%%%%%%%%%%%%%%%%%%%
 
 \addcontentsline{toc}{chapter}{Dedicatoria}
 {\bf\Large {Dedicamos esta tesis a :}}
 \vskip 1cm
\begin{quotation}
{\it Nuestros Padres que siempre nos apoyaron en todo el transcurso de nuestra formacion academica.
\vskip 1cm
Nuestros Profesores que nos guiaron en esta etapa universiaria, especialmente a nuestro asesor.
}
\end{quotation}
%%%%%%%%%%%%%%%%%%%%%%%%%%%%%%%%%%%%%%%%%%%%%%%%%%%%%%%%%%%%%%%%%%%%%%%%%%%


%%%%%%%%%%%%%%%%%%%%%%%%%%%% AGRADECIMENTOS %%%%%%%%%%%%%%%%%%%%%%
\newpage

 \addcontentsline{toc}{chapter}{Agradecimientos}
 {\bf\Large {\flushleft{Agradecimientos}}}
 \vskip 1.5cm
\begin{quotation}
Agradecemos a Dios por habernos bendecido en toda mi vida ....
{\vskip 1cm}
A nuestros profesores del Departamento de Informática, de los cuales recibimos una gran cantidad de conocimientos  . . .
\vskip 1cm
A nuestro asesor Prof. Juan Orlando Salazar Campos que siempre se mostro disponible e interesado en ayudarnos.
\vskip 1cm
 \end{quotation}
%%%%%%%%%%%%%%%%%%%%%%%%%%%%%%%%%%%%%%%%%%%%%%%%%%%%%%%%%%%%%%%%%%%%%%%%%%%


%%%%%%%%%%%%%%%%%%%%%%%%%%%% RESUMEN%%%%%%%%%%%%%%%%%%%%%%
\newpage
\begin{center}
 \addcontentsline{toc}{chapter}{Resumen}
 {\bf\LARGE Resumen}
\end{center} 
\vskip 0.5cm
\begin{quotation}

El presente trabajo tiene como objetivo el diseño de un agente inteligente que permita el reconocimiento de expresiones faciales en secuencias dinámicas de imágenes, el agente inteligente debe analizar las imágenes y mediante el uso de algoritmos permitirá la ubicación de rostros, luego de las imágenes se extraen los rasgos faciales del rostro de la persona, para identificar expresiones como pueden ser una sonrisa, un guiño, entre otros, todo esto es captado en una secuencia dinámicas de imágenes por una cámara, para luego compararlas con nuestros patrones determinados, los cuales están guardados en una base de datos, posteriormente mediante el uso de una red neuronal se realizará una comparación con las imágenes captadas de la secuencia, para que nuestro agente inteligente pueda reconocer a la persona.    

\vskip 0.3cm
\hspace*{-0.6cm}{\bf Palabras claves:} reconocimiento, expresiones faciales, agente inteligente.
\end{quotation}
%%%%%%%%%%%%%%%%%%%%%%%%%%%%%%%%%%%%%%%%%%%%%%%%%%%%%%%%%%%%%%%%%%%%%%%%%%%%%%%%%%%%


%%%%%%%%%%%%%%%%%%%%%%%%%%%%ABSTRACT%%%%%%%%%%%%%%%%%%%%%%
\newpage
\begin{center}
 \addcontentsline{toc}{chapter}{Abstract}
 {\bf\LARGE Abstract}\vskip 1.5cm
\end{center} 
\begin{quotation}

The present work aims at the design of an intelligent agent that allows the recognition of facial expressions and dynamic sequences of images, the intelligent agent must analyze the images and by using algorithms allow the location of faces, after the images are extracted The facial features of the person, to identify expressions such as a smile, a wink, among others, all this is chained and a sequence of images by a camera, then In a database, after the use of a Red neuron we make a comparison with the images captured from the sequence, so that our intelligent agent can recognize the person.


\vskip 0.3cm
\hspace*{-0.6cm}{\bf Keywords:} recognition, facial expressions, intelligent agent.
\end{quotation}
%%%%%%%%%%%%%%%%%%%%%%%%%%%%%%%%%%%%%%%%%%%%%%%%%%%%%%%%%%%%%%%%%%%%%%%%%%%%%%


%%%%%%%%%%%%%%%%%%%%%%%%%%% LISTA DE SIMBOLOS %%%%%%%%%%%%%%%%%%%%%%
\newpage
\addcontentsline{toc}{chapter}{Lista de símbolos}
 {\bf\LARGE Lista de símbolos}
 \vskip 1.5cm
Constantes: 
\begin{enumerate}
\item[(1)]$r,\overline{r} $ \hspace*{0.8cm} Indice que denota regiones.
\item[(2)] $n $ \hspace*{1.1cm} Indice de bienes finales deseados por los consumidores.
\item[(3)] ...
\vskip 3cm
\end{enumerate} 
\vskip 0.3cm
Variables:
\begin{enumerate}
\item[(5)] $ x^{r} $ \hspace*{1cm} Vector columna que denota la actividad de producción.
\item[(6)] $ u^{r} $ \hspace*{1.2cm} . . .
\end{enumerate}
